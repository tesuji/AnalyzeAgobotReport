% \pagebreak[4]
% \hspace*{1cm}
% \pagebreak[4]
% \hspace*{1cm}
% \pagebreak[4]

\chapter{Phân tích codebase}

% \ifpdf
% \graphicspath{{Chapter1/Chapter1Figs/PNG/}{Chapter1/Chapter1Figs/PDF/}{Chapter1/Chapter1Figs/}}
% \else
% \graphicspath{{Chapter1/Chapter1Figs/EPS/}{Chapter1/Chapter1Figs/}}
% \fi

\section{Cấu trúc mã nguồn}

Cấu trúc mã nguồn đề cập đến các đặc điểm thiết kế và cài đặt của mã bot.
Agobot bao gồm 20.000 dòng C++ với tính tương thích trên nhiều nền tảng và
được phát hành giấy phép mã nguồn mở GPL.

Agobot được viết bởi Ago với bí danh Wonk, thanh niên người Đức bị
bắt vào tháng 5 năm 2004 vì tội phạm mạng. Các phiên bản mới nhất có của Agobot
có thiết kế trừu tượng cấp cao với nhiều thành phần cấp cao bao gồm~\cite{inside,honeynet53}:

\begin{itemize}
\item Cơ chế điều khiển và chỉ huy dựa trên IRC
\item Lượng lớn các mã độc dùng để khai thác mục tiêu
\item Khả năng thực hiện các kiểu tấn công DoS khác nhau
\item Thành phần hỗ trợ hỗ trợ mã hóa shellcode và thành phần cơ chế làm rối cơ bản dựa trên đa hình
\item Khả năng thu thập mật khẩu Paypal, khoá phần mềm AOL và các thông tin
	nhạy cảm khác từ máy bị lây nhiễm thông qua bắt gói tin, ghi lại sự kiện
	bàn phím (key logger) hoặc tìm kiếm các khoá Windows Registry
\item Cơ chế bảo vệ và tăng quyền kiểm soát các hệ thống bị chiếm đoạt sẵn
	bằng cách đóng cái cửa sau (backdoor), vá các lổ hỗng khác hoặc vô hiệu
	truy cập vào các trang web chống virus
\item Cơ chế chống gỡ lỗi và dịch ngược (anti-debugging) bởi
	các công cụ nổi tiếng như SoftIce, Ollydbg, v.v.
	Cơ chế nhận diện ảo hoá trong máy ảo (ví dụ: VMWare và Virtual PC).
\end{itemize}

Agobot có kiến trúc nguyên khối, thể hiện tính sáng tạo trong thiết kế và
tuân thủ nguyên tắc thiết kế với tập các cấu trúc và kiểu dữ liệu mở rộng
được kiểm soát chặt chẽ và nguyên tắc kỹ thuật phần mềm thông qua tính
mô đun hoá và tài liệu mã nguồn mạnh mẽ.

Agobot được cấu trúc theo cách rất mô đun và rất dễ dàng để thêm các lệnh
hoặc máy quét cho các lỗ hổng khác: Chỉ bằng cách cần mở rộng lớp
\texttt{CCommandHandler} hoặc \texttt{CScanner} và thêm tính năng tương ứng.
Một mẹo nhỏ \cite{honeynetgrep}: Chỉ cần dùng lệnh grep mã nguồn với từ khoá \texttt{RegisterCommand},
ta sẽ có được toàn bộ danh sách lệnh với một mô tả đầy đủ của tất cả các tính năng.

\section{Cơ chế điều khiển từ xa}
\section{Điều khiển máy chủ}
\section{Cơ chế lan truyền}
\section{Phương thức khai thác và tấn công}
\section{Phương thức phân phối malware}
\section{Phương thức obfuscation}
\section{Phương thức deception}

\section{Kết chương}
Viết luận văn bằng