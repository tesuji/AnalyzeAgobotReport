
% Danh mục thuật ngữ
\newglossaryentry{codebase}
{
	name={codebase},
	description={toàn bộ tập mã nguồn được sử dụng để xây dựng một hệ thống phần mềm, ứng dụng hoặc thành phần phần mềm cụ thể},
	user1={"https://en.wikipedia.org/wiki/Codebase"}
}

\newglossaryentry{malware}
{
	name={malware},
	description={phần mềm ác tính, phần mềm độc hại, phần mềm gây hại hay mã độc},
	user1={"https://vi.wikipedia.org/wiki/Ph\%E1\%BA\%A7n_m\%E1\%BB\%81m_\%C3\%A1c_\%C3\%BD"}
}

\newglossaryentry{obfuscation}
{
	name={làm rối},
	description={làm rối mã nguồn để các công cụ tự động không phát hiện phần mềm độc hại},
	user1={"https://en.wikipedia.org/wiki/Obfuscation_(software)"}
}

\newglossaryentry{deception}
{
	name={tránh phát hiện},
	description={cơ chế được sử dụng để tránh phát hiện khi bot được cài đặt trên một máy chủ}
}

%% Danh mục từ viết tắt
%% Ví dụ
%\newacronym{http}{HTTP}{HyperText Transfer Protocol}
%\newacronym{ai}{AI}{Artificial Intelligence}
%\newacronym{cart}{CART}{Classification and Regression Trees}
%\newacronym{csdl}{CSDL}{Cơ Sở Dữ Liệu}
%\newacronym{xss}{XSS}{Cross-Site Scripting}
%\newacronym{id3}{ID3}{Iterative Dichotomiser 3}
%\newacronym{csic}{CSIC}{Consejo Superior de Investigaciones Científicas}
